\usection{Lecture 8: Evolutionary Game Theory}
\newsection
\subsection*{Logistics}
\begin{itemize}
    \item HW2 posted, due one week from today.
\end{itemize}

Evolutionary Game Theory dates back to 1973 (John Maynard Smith, George Price).

The idea is that we have a large population of potential players. For instance, different species/phenotypes of individuals. We pick two agents from this population to play a game. The winner reproduces faster. 

\definition{Evolutionary game theory terminology}{
    Let $G=(\{1,2\},\{S_1,S_2\},{\pi_1,\pi_2})$ be a finite, symmetric game (i.e. $S_1=S_2=S, \pi_1=\pi_2=\pi$). The \textbf{fitness} of a player playing strategy $s_i\in S$ is the expected payoff against a random member of the population.\\
    A strategy $s_j$ is said to \textbf{invade} strategy $s_i$ at level $\epsilon\in[0,1]$ if $\epsilon$ of the population is type $j$ and $1-\epsilon$ of the population is type $i$.\\
    An \textbf{evolutionary stable strategy} (ESS) is a strategy $s_i$ if $\exists\bar\epsilon >0$ such that if any other strategy $s_j$ invades at level $\epsilon<\bar\epsilon$, the fitness of type $i<$ the fitness of type $j$.
}
\begin{remark}
    The fitness depends on the distribution of strategy `types'.
\end{remark}
\begin{remark}
    Intuitively, an evolutionary stable strategy can withstand invation of another strategy up to $\bar\epsilon$.
\end{remark}
\begin{aexample}{}{}
    Consider this payoff for beetle sizes. \begin{center}
        \begin{tabular}{|c|c c|} 
            \hline &smol& beeg \\
            
            \hline
            smol&(5,5)&(1,8)\\
            \hline beeg&(8,1)&(3,3)
            \\ \hline
        \end{tabular}
    \end{center}
    Two smol beetles share the food. Beeg beetle takes most of the food. Two beeg beetles fight too much and the food is not as useful.

    With only smol beetles, the fitness is $5$. Now let us introduce $\epsilon>0$ beeg beetles into the population. The smol beetles now have payoff \[
    5(1-\epsilon)+ \epsilon = 5-4\epsilon.
    \]
    But the beeg beetles have payoff \[
    8(1-\epsilon) + 3\epsilon = 8-5\epsilon > 5-4\epsilon
    \]
    for $0\leq\epsilon\leq 1$. Therefore, being smol is not an evolutionary stable strategy.

    We change variables $\epsilon\mapsto 1-\epsilon$ to see that being beeg is an evolutionary stable strategy.
\end{aexample}
\proposition{
    An evolutionary stationary strategy is a Nash Equilibrium.
}
\begin{proof}
    Suppose a strategy $s^*$ is an evolutionary stable strategy, then \[
        (1-\epsilon)\pi(s^*,s^*)+\epsilon \pi(s^*,s')>
        (1-\epsilon)\pi(s',s^*)+\epsilon\pi(s',s')
        \]
        for some $\epsilon$ small enough and all $s'\neq s^*$.
        Now we take $\epsilon\to 0 $,\[
            \pi(s^*,s^*)\geq\pi(s',s^*).
        \]
\end{proof}
\begin{remark}
    If the Nash equilibrium is weak, then the other inequality is strict:\[
        \pi(s^*,s')>
       \pi(s',s')
    \]
\end{remark}

The converse is also true, a strict nash equilibrium is an evolutionary stable strategy. \[
\textrm{Strict Nash} \subset \textrm{ESS} \subset {Nash}
\]
These are distinct sets.\begin{aexample}{ESS but not strict}{}
    \begin{center}
        \begin{tabular}{|c|c c|} 
            \hline &A& B \\
            
            \hline
            A&(0,0)&(6,0)\\
            \hline B&(0,6)&(3,3)
            \\ \hline
        \end{tabular}
    \end{center}
    There is a weak Nash equilibirum (A,A). But this is an evolutionary stable strategy. \[
    (1-\epsilon)(0)+\epsilon(6)> (1-\epsilon)(0) + \epsilon(3).
    \]
    The reasoning is that adding in $B$ benefits $A$.
\end{aexample}
\begin{aexample}{Nash but not ESS}{}
    \begin{center}
        \begin{tabular}{|c|c c|} 
            \hline &A& B \\
            
            \hline
            A&(4,4)&(0,4)\\
            \hline B&(4,0)&(3,3)
            \\ \hline
        \end{tabular}
    \end{center}
    There is a weak Nash equilibirum (A,A). But this is an not an evolutionary stable strategy. \[
    (1-\epsilon)(4)+\epsilon(0)<(1-\epsilon)(4) + \epsilon(3).
    \]
    Adding in any small amount of B to the population of $A$ causes a decrease in payoff.
\end{aexample}
\definition{Mixed evolutionary stable strategy}{
    A mixed strategy $\sigma^*$ is an ESS if there exists $\bar\epsilon > 0$ such that for all $\epsilon<\bar\epsilon$,\[
        \pi_i(\sigma^*,\epsilon\sigma'+(1-\epsilon)\sigma^*)>\pi_i(\sigma',\epsilon\sigma'+(1-\epsilon)\sigma^*),
    \]
    for any other mixed strategy $\sigma'$.
}
\begin{aexample}{Hawk Dove}{hawkdove}
    \begin{center}
        \begin{tabular}{|c|c c|} 
            \hline &D& H \\
            
            \hline
            D&(3,3)&(1,5)\\
            \hline H&(5,1)&(0,0)
            \\ \hline
        \end{tabular}
    \end{center}

    There are two symmetric strong Nash equilibria (D,H) and (H,D). However, these are not candidates for a pure ESS. We find the mixed ESS by letting a strategy $p$ chance of playing $D$ and $(1-p)$ chance of playing $H$. We solve for \[
    p(3)+(1-p)(1)= p(5)+(1-p)0 \implies p=\frac{1}{3}.
    \]
    This is the only candidate for an evolutionary stable strategy. Indeed,for any other strategy playing $(1,1-q)$ invading $(1/3,2/3)$ at level $\epsilon$, the probability of meeting a dove strategy is \[
    r=(1-\epsilon)\frac{1}{3}+\epsilon q
    \]
    The payoffs are \[
        (\pi_i(\sigma^*,\epsilon\sigma'+(1-\epsilon)\sigma^*))-(\pi_i(\sigma',\epsilon\sigma'+(1-\epsilon)\sigma^*)) = \frac{1}{3}r3 + \frac{1}{3}(1-r)1+\frac{2}{3}r3+\frac{2}{3}(1-r)0 -\ldots >0
    \] Just trust me bro. 
\end{aexample}
By the linearity of expectations, \begin{align*}
    &(\pi_i(\sigma^*,\epsilon\sigma'+(1-\epsilon)\sigma^*))-(\pi_i(\sigma',\epsilon\sigma'+(1-\epsilon)\sigma^*)) \\=& \epsilon(\pi(\sigma^*,\sigma')-\pi(\sigma^*,\sigma^*)) + (1-\epsilon)\epsilon(\pi(\sigma',\sigma')-\pi(\sigma',\sigma^*)) 
\end{align*}
Assume that $\sigma^*$ plays all actions with some probability, and that $\sigma^*$ is a Nash equilibrium. Then \[
\pi(\sigma^*,\sigma^*)=\pi(\sigma,\sigma^*),
\]
as a different mixing of the same strategies. Therefore, it suffices to check that \[
    \pi(\sigma^*,\sigma')-\pi(\sigma^*,\sigma^*)>0
\]to see that it is an ESS.

\subsection*{Replicator Dynamics}
\definition[]{Replicator Dynamics}{
    Let $x_i(t)$ be the fraction of type $i$ in the population at time $t$, so that $\sum_i x_i(t)=1$ for all $t$.
    Let $f_i(\bar{x}(t))$ be the fitness of type $i$ at time $t$.
    Under \textbf{replicator dyanmics}, we evolve the system through the differential equation\begin{align*}
        \frac{d}{dt}\frac{x_i}{x_j}=\frac{x_i}{x_j}[f_i(\bar{x}(t))-f_j(\bar{x}(t))].
    \end{align*}
}
\proposition{
Under replicator dynamics,\[
\frac{d}{dt}x_i(t) = x_i(t)[f_i(\bar{x}(t))-\phi(\bar{x}(t))],
\]where\[
\phi(\bar{x}(t))\defeq \sum_j x_j(t) f_j(\bar{x}(t))
\]
is the average fitness of the population at time $t$.
}
\theorem[]{}{
    If $x^*$ is an ESS, then it is asymptotically stable for replicator dynamics.
}